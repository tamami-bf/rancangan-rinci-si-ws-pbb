\documentclass[pdftex,12pt, oneside]{article}

%\usepackage[paperwidth=8.5in, paperheight=13in]{geometry} % Folio
\usepackage[paperwidth=8.27in, paperheight=11.69in]{geometry} % A4

\usepackage{makeidx}         % allows index generation
\usepackage{graphicx}        % standard LaTeX graphics tool
                             % when including figure files
\usepackage[bottom]{footmisc}% places footnotes at page bottom
\usepackage[english]{babel}
\usepackage{enumerate}
\usepackage{paralist}
\usepackage{float}
\usepackage{gensymb}  
\usepackage{listings}
\usepackage{color}
\usepackage{mathtools} % atau \usepackage{amsmath}
\renewcommand{\baselinestretch}{1.5}

\newcommand{\HRule}{\rule{\linewidth}{0.5mm}}

\definecolor{codegreen}{rgb}{0,0.6,0}
\definecolor{codegray}{rgb}{0.5,0.5,0.5}
\definecolor{codepurple}{rgb}{0.58,0,0.82}
\definecolor{backcolor}{rgb}{0.95,0.95,0.92}

\lstdefinestyle{mystyle}{
  backgroundcolor=\color{backcolor},
  commentstyle=\color{codegreen},
  keywordstyle=\color{magenta},
  stringstyle=\color{codepurple},
  basicstyle=\footnotesize,
  breakatwhitespace=false,
  breaklines=true,
  captionpos=b,
  keepspaces=true,
  numbers=left,
  numbersep=5pt,
  showspaces=false,
  showstringspaces=false,
  showtabs=false,
  tabsize=2
}

\lstset{style=mystyle}


\begin{document}
\sloppy % biar section ga melebar melewati kertas

\begin{center}
{\large RANCANGAN RINCI SISTEM \textit{WEB SERVICES} SEBAGAI CARA KOMUNIKASI DENGAN TEMPAT PEMBAYARAN DALAM PENCATATAN PEMBAYARAN PAJAK BUMI DAN BANGUNAN PERDESAAN DAN PERKOTAAN DI KABUPATEN BREBES.}
\\[1cm]
DD MMM 2016\\
Priyanto Tamami, S.Kom.
\end{center}

%\frontmatter%%%%%%%%%%%%%%%%%%%%%%%%%%%%%%%%%%%%%%%%%%%%%%%%%%%%%%


%%%%%%%%%%%%%%%%%%%%%%%%%%%%%%%%%%%%%%%%%%%%%%%%%%%%%%%%%%%%%%%%%%%%%%

\section{SISTEM KOMPUTER}

Sistem komputer yang digunakan akan terbagi menjadi 3 (tiga) yaitu :

\begin{enumerate}[1.]
  \item Sebagai \textit{server} basis data adalah sebagai berikut :

    \begin{itemize}
      \item Prosesor Intel Xeon 2,4GHz
      \item Memori 44GB
      \item Sistem Operasi Windows Server 2008 R2 64 bit.
    \end{itemize}
    
  \item Sebagai \textit{server} aplikasi adalah sebagai berikut :
  
    \begin{itemize}
      \item Prosesor Intel Xeon 3,1GHz
      \item Memori 4GB
      \item Sistem Operasi CentOS 6.2 64 bit
    \end{itemize}
    
  \item Sebagai \textit{client} spesifikasi yang digunakan bebas, dapat menggunakan sistem komputer apapun yang dapat berkomunikasi melalui jaringan TCP/IP.
  
  Karena \textit{client} nantinya adalah Bank sebagai tempat pembayaran, maka sebagai sarana untuk uji coba dapat menggunakan sistem komputer apapun dengan \textit{browser} Chrome / Firefox.

\end{enumerate}

\section{SISTEM JARINGAN}

Sistem jaringan yang nantinya dibangun akan terlihat seperti pada gambar \ref{fig:network-diagram} :

\begin{figure}[H]
  \centering
  \includegraphics[width=0.5\textwidth]{./resources/diagram/network-diagram}
  \caption{Diagram Sistem Jaringan \textit{Web Services} PBB}
  \label{fig:network-diagram}
\end{figure}

Dari diagram tersebut, gambar awan adalah simbol untuk jaringan internet. Untuk melakukan akses ke \textit{server web service} akan melalui modem dibawahnya, kemudian akan menghubungi VPN \textit{server} terlebih dahulu untuk mendapatkan otentikasi atau akses ke dalam jaringan internal.

Setelah sukses melakukan otentikasi ke \textit{server} VPN, selanjutnya \textit{client} dalam hal ini Bank akan melakukan akses ke \textit{server web service} langsung, dimana \textit{server web service} akan melakukan komunikasi dengan \textit{server} basis data.

\section{SISTEM BASIS DATA}



\section{PROSEDUR AKTIVITAS}

% isinya :
%   - diagram use-case
%   - activity diagram
%   - class diagram
%   - entity relational diagram (db)
%   - sequence diagram
%   - deployment diagram

\section{SUMBER DAYA MANUSIA}

\end{document}